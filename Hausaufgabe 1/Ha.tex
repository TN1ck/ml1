\documentclass[10pt,a4paper]{article}
\usepackage[utf8]{inputenc}
\usepackage{amsmath}
\usepackage{amsfonts}
\usepackage{amssymb}
\setlength{\parindent}{0cm}
\usepackage{setspace}
\usepackage{mathpazo}
\usepackage{graphicx}
\usepackage{wasysym} 
\usepackage{booktabs}
\usepackage{verbatim}
\usepackage{microtype}
\usepackage{siunitx}
\usepackage{cleveref}
\usepackage[colorlinks=false, pdfborder={0 0 0}]{hyperref}
\usepackage{paralist}
\usepackage{pst-all}
\usepackage{pstricks}
\usepackage{pst-node}
\usepackage{tikz}
\usepackage{tkz-berge}
\usetikzlibrary{trees,petri,decorations,arrows,automata,shapes,shadows,positioning,plotmarks}
\usepackage[a4paper,
left=3.0cm, right=3.0cm,
top=2.0cm, bottom=2.0cm]{geometry}
\usepackage{fullpage}
\usepackage[german]{babel}
%\usepackage{pst-all}
%\usepackage{pstricks}
\setlength{\unitlength}{1cm}
\newcommand{\N}{\mathbb{N}}
\newcommand{\A}{\mathcal{A}}
\newcommand{\R}{\mathbb{R}}
\newcommand\numberthis{\addtocounter{equation}{1}\tag{\theequation}}
\usepackage{mathpazo}

\author{Tom Nick, Niklas Gebauer, Felix Bohmann}
\title{Machine Learning}
\begin{document}
\begin{center}
\Large{\textsc{Machine Learning 1: Assignment 1}} \\
\end{center}

\begin{tabbing}
Tom Nick \hspace{0.9cm}\= 340528\\
Niklas Gebauer\>  xxxxx \\
Felix Bohmann\> xxxxxx
\end{tabbing}

\subsection*{Exercise 1}
\subsubsection*{a)}
\begin{align}
	P(error) &= \int P(error\mid x)p(x) dx \label{eq1} \\
	P(error\mid x) &= \min (P(w_1\mid x), P(w_2\mid x)) \label{eq2}
\end{align}

With these equations, we want to show that

\begin{align}
	P(error) \le \int \frac{2}{\frac{1}{P(w_1\mid x)} + \frac{1}{P(w_2 \mid x)}} p(x) dx \label{eq3}
\end{align}

At first, without restricting the general case, we assume that $P(w_1 \mid x) \ge P(w_2 \mid x) \label{ass1}$, that is the function $P(error \mid x) = P(w_2 \mid x)$. Now with \eqref{eq1}, \eqref{eq2} and \eqref{eq3} we have:

\begin{align*}
	\int P(w_2 \mid x)p(x) dx \le \int \frac{2}{\frac{1}{P(w_1\mid x)} + \frac{1}{P(w_2 \mid x)}} p(x) dx
\end{align*}

Because both sides are integrating over the same variable we can simplify the term to:
\begin{align*}
	&P(w_2 \mid x)p(x) \le \frac{2}{\frac{1}{P(w_1\mid x)} + \frac{1}{P(w_2 \mid x)}} p(x) \\
	&\Leftrightarrow \frac{1}{P(w_2 \mid x)p(x)} \ge \frac{\frac{1}{P(w_1\mid x)} + \frac{1}{P(w_2 \mid x)}}{2} p(x) \\
	&\Leftrightarrow \frac{1}{P(w_2 \mid x)} \ge \frac{\frac{1}{P(w_1\mid x)} + \frac{1}{P(w_2 \mid x)}}{2} \\
	&\Leftrightarrow \frac{2}{P(w_2 \mid x)} \ge \frac{1}{P(w_1\mid x)} + \frac{1}{P(w_2 \mid x)}  \\
	&\Leftrightarrow \frac{1}{P(w_2 \mid x)} \ge \frac{1}{P(w_1\mid x)} \\
	&\Leftrightarrow P(w_2 \mid x) \le P(w_1\mid x)
\end{align*}

This holds true with the assumptions \ref{ass1} we made earlier.

\subsubsection*{b)}

With this result, we now show that:

\begin{align*}
P(error) \le \frac{2P(w_1)P(w_2)}{\sqrt{P(w_1)^2 + (4\mu^2 + 2)P(w_1)P(w_2) + P(w_2)^2}}
\end{align*}

While using the univariate probability distribution:
$$p(x\mid w_1) = \frac{\pi^{-1}}{1+ (x - \mu)^2} \textrm{ and } p(x \mid w_2) = \frac{\pi^{-1}}{1 + (x + \mu)^2}$$

With the rule of bayes we have $P(w_1 \mid x) = \frac{p(x \mid w_1)P(w_1)}{p(x)}$:

\begin{align}
	P(error) &\le \int \frac{2}{\frac{1}{P(w_1\mid x)} + \frac{1}{P(w_2 \mid x)}} p(x) dx \\
	\Leftrightarrow P(error) &\le \int \frac{2}{\frac{1}{ \frac{p(x \mid w_1)P(w_1)}{p(x)} } + \frac{1}{\frac{p(x \mid w_2)P(w_2)}{p(x)}}} p(x) dx \\
	\Leftrightarrow P(error) &\le \int \frac{2}{\frac{p(x)}{p(x \mid w_1)P(w_1)} + \frac{p(x)}{p(x \mid w_2)P(w_2)}} p(x) dx \\
	\Leftrightarrow P(error) &\le \int \frac{2}{\frac{1}{p(x \mid w_1)P(w_1)} + \frac{1}{p(x \mid w_2)P(w_2)}}dx \\
	\Leftrightarrow P(error) &\le \int \frac{2}{\frac{1}{ \frac{\pi^{-1}}{1+ (x - \mu)^2} P(w_1)} + \frac{1}{ \frac{\pi^{-1}}{1 + (x + \mu)^2} P(w_2)}}dx \\
	\Leftrightarrow P(error) &\le \int \frac{2}{\frac{1}{ \frac{\pi^{-1} P(w_1)}{1+ (x - \mu)^2}} + \frac{1}{ \frac{\pi^{-1} P(w_2)}{1 + (x + \mu)^2}}}dx \\
	\Leftrightarrow P(error) &\le \int \frac{2}{\frac{1+ (x - \mu)^2}{\pi^{-1} P(w_1)} +  \frac{1 + (x + \mu)^2}{\pi^{-1} P(w_2)}}dx \\
\end{align}







\end{document}