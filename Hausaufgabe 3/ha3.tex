\documentclass[10pt,a4paper]{article}
\usepackage[utf8]{inputenc}
\usepackage{amsmath}
\usepackage{amsfonts}
\usepackage{amssymb}
\setlength{\parindent}{0cm}
\usepackage{setspace}
\usepackage{mathpazo}
\usepackage{graphicx}
\usepackage{wasysym} 
\usepackage{booktabs}
\usepackage{enumerate}
\usepackage{verbatim}
\usepackage{microtype}
\usepackage{siunitx}
\usepackage{cleveref}
\usepackage[colorlinks=false, pdfborder={0 0 0}]{hyperref}
\usepackage{paralist}
\usepackage{pst-all}
\usepackage{pstricks}
\usepackage{pst-node}
\usepackage{tikz}
\usepackage{tkz-berge}
\usetikzlibrary{trees,petri,decorations,arrows,automata,shapes,shadows,positioning,plotmarks}
\usepackage[a4paper,
left=3.0cm, right=3.0cm,
top=2.0cm, bottom=2.0cm]{geometry}
\usepackage{fullpage}
\usepackage[german]{babel}
%\usepackage{pst-all}
%\usepackage{pstricks}
\setlength{\unitlength}{1cm}
\newcommand{\N}{\mathbb{N}}
\newcommand{\A}{\mathcal{A}}
\newcommand{\R}{\mathbb{R}}
\newcommand{\D}{\mathcal{D}}
\newcommand{\tr}{\text{tr}}
\newcommand\numberthis{\addtocounter{equation}{1}\tag{\theequation}}
\usepackage{mathpazo}

\author{Tom Nick, Niklas Gebauer}
\title{Machine Learning}
\begin{document}
\begin{center}
\Large{\textsc{Machine Learning 1: Assignment 3}} \\
\end{center}

\begin{tabbing}
Tom Nick \hspace{0.9cm}\= 340528\\
Niklas Gebauer\>  340942 
\end{tabbing}

\subsection*{Exercise 1}
\begin{enumerate}[(a)]
\item
In the following we want to minimize the objective function
\begin{align*}
    J(\theta) = \sum_{k = i}^{n} \|\theta-x_k\|^2
\end{align*}
subject to the constraint $\theta^T b = 0$, where $x_1, ..., x_k, \theta, b \in \mathbb{R}^d$.

Therefore we will define the Lagrangian function and set its gradient to zero in order to solve for $\lambda$ and $\theta$:

\begin{align*}
    \mathcal{L}(\theta,\lambda) &= J(\theta) + \lambda\theta^T b = \sum_{k = i}^{n} \|\theta-x_k\|^2 + \lambda\theta^T b \\
    \nabla\mathcal{L} &= \left(
        \begin{array}{c}
            [\sum_{k = i}^{n} 2(\theta-x_k)] + \lambda b\\
            \theta^T b\\
        \end{array}
    \right) \overset{!}{=} \vec{0} \\
\end{align*}
We will at first solve the first entry of our gradient for $\theta$:
\begin{align*}
    [\sum_{k = i}^{n} 2(\theta-x_k)] + \lambda b &= 0 \\
    \Leftrightarrow 2n\theta-2[\sum_{k = i}^{n}x_k] + \lambda b &= 0\\
    \Leftrightarrow 2[\sum_{k = i}^{n}x_k] - \lambda b &= 2n\theta\\
    \Leftrightarrow \frac{1}{n}[\sum_{k = i}^{n}x_k] - \frac{1}{2n}\lambda b &= \theta
\end{align*}
Let's plug this into the second entry to solve for $\lambda$:
\begin{align*}
    \theta^T b &= 0 \\
    \Leftrightarrow (\frac{1}{n}[\sum_{k = i}^{n}x_k] - \frac{\lambda}{2n}b)^T b &= 0\\
    \Leftrightarrow (\frac{1}{n}[\sum_{k = i}^{n}x_k^T] - \frac{\lambda}{2n} b^T) b &= 0\\
    \Leftrightarrow \frac{1}{n}[\sum_{k = i}^{n}x_k^T]b - \frac{\lambda}{2n} b^T b &= 0\\    
    \Leftrightarrow \frac{1}{n}[\sum_{k = i}^{n}x_k^T]b &= \frac{\lambda}{2n} b^T b\\ 
    \Leftrightarrow \frac{2}{b^T b}[\sum_{k = i}^{n}x_k^T]b &= \lambda
\end{align*}
Finally, we can plug this $\lambda$ into our formula describing $\theta$, yielding the result:
\begin{align*}
    \theta &= \frac{1}{n}[\sum_{k = i}^{n}x_k] - \frac{1}{2n}\lambda b \\
    \Leftrightarrow \theta &= \frac{1}{n}[\sum_{k = i}^{n}x_k] - \frac{1}{2n}(\frac{2}{b^T b}[\sum_{k = i}^{n}x_k^T]b) b \\
    \Leftrightarrow \theta &= \frac{1}{n}([\sum_{k = i}^{n}x_k] - \frac{[\sum_{k = i}^{n}x_k^Tb]}{b^T b} b) \\
\end{align*}

Similar to the minimal solution for $\theta$ of the unconstrained objective, we again find the the empirical mean in our solution. But this time we subtract a scaled version of the vector $b$. This means, that the solution still has to lie close to the empirical mean. If we imagine a plot with contour lines of $J(\theta)$ there has to be a 'valley' which is the empirical mean and a line cutting the surface which specifies the points where our constraint holds. The second term of our solution will move $\theta$ from total minimum (the empirical mean) to the point where the line lies deepest in the plane and where the lines and $J$'s gradient both are perpendicular to the line. So we have the minimal solution that still fulfills the constraint.

\item
Now we will repeat the same procedure for the objective above with a different constraint

($\|\theta-c\|^2 = 1$, $c \in \mathbb{R}^d$):
\begin{align*}
    \mathcal{L}(\theta,\lambda) &= J(\theta) + \lambda \|\theta - c\|^2 - \lambda \\
    \nabla\mathcal{L} &= \left(
        \begin{array}{c}
            [\sum_{k = i}^{n} 2(\theta-x_k)] + \lambda 2(\theta - c)\\
            \|\theta - c\|^2 - 1\\
        \end{array}
    \right) \overset{!}{=} \vec{0} \\
\end{align*}
Solve the first entry for $\theta$:
\begin{align*}
    [\sum_{k = i}^{n} 2(\theta-x_k)] + \lambda 2(\theta - c) &= 0 \\
    \Leftrightarrow 2n\theta - 2 [\sum_{k = i}^{n}x_k] + 2\lambda \theta - 2\lambda c &= 0 \\
    \Leftrightarrow 2n\theta + 2\lambda \theta &= 2 [\sum_{k = i}^{n}x_k] + 2\lambda c \\
    \Leftrightarrow \theta = \frac{1}{n+\lambda} ([\sum_{k = i}^{n}x_k] + \lambda c)\\ 
\end{align*}
Let's plug this into the second entry to solve for $\lambda$:
\begin{align*}
    \|\theta - c\|^2 - 1 &= 0 \\
    \Leftrightarrow \| \frac{1}{n+\lambda} ([\sum_{k = i}^{n}x_k] + \lambda c) - c\|^2 - 1 &= 0\\
    \Leftrightarrow \| \frac{1}{n+\lambda} [\sum_{k = i}^{n}x_k] + \frac{\lambda}{n+\lambda} c - \frac{n+\lambda}{n+\lambda}c\|^2 - 1 &= 0\\
    \Leftrightarrow \| \frac{1}{n+\lambda} [\sum_{k = i}^{n}x_k] - \frac{n}{n+\lambda} c\|^2 &= 1\\
    \Leftrightarrow \| \frac{1}{n+\lambda} ([\sum_{k = i}^{n}x_k] - n c)\|^2 &= 1\\
    \Leftrightarrow  (\frac{1}{n+\lambda} ([\sum_{k = i}^{n}x_k] - n c))^T (\frac{1}{n+\lambda} ([\sum_{k = i}^{n}x_k] - n c)) &= 1\\
    \Leftrightarrow \frac{1}{(n+\lambda)^2}([\sum_{k = i}^{n}x_k] - n c)^T ([\sum_{k = i}^{n}x_k] - n c) &= 1\\
    \Leftrightarrow \|[\sum_{k = i}^{n}x_k - c]\|^2 &= (\lambda + n)^2\\
    \Leftrightarrow \|[\sum_{k = i}^{n}x_k - c]\|^2 &= \lambda^2 + 2n\lambda +n^2\\
    \Leftrightarrow 0 &= \lambda^2 + 2n\lambda + n^2 - \|[\sum_{k = i}^{n}x_k - c]\|^2 \\
    \overset{pq-formula}{\Rightarrow } \frac{-2n}{2} \pm \sqrt{\frac{2n}{2}^2-(n^2 - \|[\sum_{k = i}^{n}x_k - c]\|^2)} &= \lambda_{1,2}\\
    \Leftrightarrow -n \pm \sqrt{\|[\sum_{k = i}^{n}x_k - c]\|^2} &= \lambda_{1,2}\\
    \Rightarrow -n + \|[\sum_{k = i}^{n}x_k - c]\| &= \lambda_1\\
    \Rightarrow-n -  \|[\sum_{k = i}^{n}x_k - c]\| &= \lambda_2
\end{align*}
We can now compute two solutions for $\theta$ using $\lambda_1$ and $\lambda_2$:\\ \\
$\lambda_1:\\$
\begin{align*}
	\theta_1 &= \frac{1}{n+\lambda_1} ([\sum_{k = i}^{n}x_k] + \lambda_1 c)\\
	&=  \frac{1}{n+( -n + \|[\sum_{k = i}^{n}x_k - c]\|)} ([\sum_{k = i}^{n}x_k] +( -n + \|[\sum_{k = i}^{n}x_k - c]\|) c)\\
	&= \frac{1}{\|[\sum_{k = i}^{n}x_k - c]\|}  ([\sum_{k = i}^{n}x_k] -nc + \|[\sum_{k = i}^{n}x_k - c]\| c)\\
	&=  \frac{1}{\|[\sum_{k = i}^{n}x_k - c]\|} [\sum_{k = i}^{n}x_k-c] + c\\
\end{align*}\\ \\
$\lambda_2:\\$
\begin{align*}
	\theta_2 &= \frac{1}{n+\lambda_2} ([\sum_{k = i}^{n}x_k] + \lambda_2 c)\\
	&=  \frac{1}{n+( -n - \|[\sum_{k = i}^{n}x_k - c]\|)} ([\sum_{k = i}^{n}x_k] +( -n - \|[\sum_{k = i}^{n}x_k - c]\|) c)\\
	&= -\frac{1}{\|[\sum_{k = i}^{n}x_k - c]\|}  ([\sum_{k = i}^{n}x_k] -nc - \|[\sum_{k = i}^{n}x_k - c]\| c)\\
	&=  \frac{1}{\|[\sum_{k = i}^{n}x_k - c]\|} [\sum_{k = i}^{n}c-x_k] + c\\
\end{align*}
This time we again see something similar to the mean, but $\theta$ is dependend on $c$. In both formulas we have a normalized vector (length = 1) and add up c, so that the constraint holds (if we subtract c again we will have a vector thatstill has length one). Since we know from the objective without constraint that the empirical mean is the minimum, it's clear that $\theta_1$ is the minimizing parameter: it stays close to the (normalized mean) but every data point gets a little offset of -c so that the latter addition of c doesn't take the result too far away from the empirical mean.

$\theta_2$ is rather close to c since it centers all the points around c. Thus it won't yield as small results in our objective function as $\theta_1$, which is closer to the empirical mean.

\end{enumerate}

\subsection*{Exercise 2}
The Scatter-matrix $S$ is defined as $$\sum_{k = 1}^{n} (x_k - m)(x_k - m)^T$$ where $m$ is defined as the mean $$m = \frac{1}{n}\sum_{k=1}^{n}x_k$$. 
\begin{enumerate}[(a)]
\item
It is to show, that the trace $\tr$ of $S$ is an upper bound for $\lambda_1$ which denotes the highest eigenvalue of $S$, the trace of S is defined as $\sum^d_{i=1} S_{ii}$.

We use four facts to prove this:
\begin{enumerate}[1.]
    \item \textbf{$S$ is symmetric} \\
    It is easy to see, that $S$ is a symmetric matrix - a vector multiplied by its transposed self creates always a symmetric matrix ($AA^T = (AA^T)^T \Leftrightarrow (A^T)^TA^T \Leftrightarrow AA^T$), $(x_k + m)(x_k + m)$ results in a symmetric matrix therefore as well. The sum of two symmetric matrices results in another symmetric matrix, $S$ is therefore symmetric.
    \item \textbf{$S$ is positive semidefinite}\\
    For each vector $a \in \mathbb{R}^d, a^TSa \ge 0$:\\
    \begin{align*}
        a^TSa &= a^T(\sum_{k = 1}^{n} (x_k - m)(x_k - m)^T)a\\
        &= \sum_{k = 1}^{n} a^T(x_k - m)(x_k - m)^Ta\\
        &= \sum_{k = 1}^{n} ((x_k - m)^Ta)^T((x_k - m)^Ta)\\
        &= \sum_{k = 1}^{n} ((x_k - m)^Ta)^2
    \end{align*}
    \item \textbf{Thus $S$ can be decomposed into $Q\Lambda Q^T$ where $\Lambda$ is a diagonal with the eigevalues of $S$ as values.} %thought it should be the eigenvalues of S instead of the eigenvalues of A..?
    \item $\tr(AQA^T) = \tr(Q)$ when $A$ is orthogonal \\
    $\tr(AQA^T) = \tr((AQ)A^T) = \tr(A^T(AQ)) = \tr(Q)$
Since $((x_k - m)^Ta)^2 \ge 0$ it holds that $a^TSa \ge 0$.
\end{enumerate}
With 3. and 4. we get:
\begin{align*}
\tr(S) = \tr(Q\Lambda Q^T) = \tr(\Lambda) = \sum_{k = 1}^{d} \lambda_k
\end{align*}
The trace of $S$ is therefore the sum of all its eigenvalues, which is trivially an upper-bound for the largest eigenvalue $\lambda_1$, if we consider 4: Since S is positive semidefinite, all of its eigenvalues are positive. So the largest eigenvalue $\lambda_1$ can't be larger than $tr(S) = \sum^d_{i=1} S_{ii}$.
\item As the upper bound is the sum of all eigenvalues, it is tighter the smaller the eigenvalues $\lambda_2 \dots \lambda_d$ relativly to the eigenvalue $\lambda_1$ are. Especially it is the same if all other eigenvalues are zero. This means for the data, that only one feature is really significant for the variance of the data. If all entries for the other features are identical, the bound will be tight.
\item
The scatter Matrix is nothing else than the covariance-matrix ($Cov(X)$) times $n$. We are now using the eigenvalue-decomposition of both matrices to show, that the eigenvalues of $S$, are the ones of the covariance-matrix scaled by $n$:
$$S = n \cdot Cov(X) = n \cdot Q\Lambda Q^T = Q(n \cdot \Lambda)Q^T = Q\overline{\Lambda}Q^T$$
It is known that the diagonal-elements of the covariance-matrix are equal to the variance of the respective dimension of $X$.

The Eigenvector of the largest eigenvalue of $Cov(X)$ points in the direction of the biggest Variance of $X$ and is also equal to the Variance of this direction. This means that the Variances along the diagonal of $Cov(X)$ are always smaller or equal to the largest eigenvalue of $Cov(X)$.

Because the values of the diagonal of $S$ and its eigenvalues are the same of $Cov(X)$, just scaled by $n$ (see above), the same thing is true for $S$. Therefore, the maximum element the diagonal of $S$ is a lower bound of $\lambda_1$.
\item
The values of the diagonal are proportional to the variances of each dimension of $X$. The more the eigenevector of $\lambda_1$ points directly along one axis - i.e. the more the maximum variance of $X$ lies in one dimension, the tighter the bound will be.
The best case, i.e. whet the lower-bound matches $\lambda_1$ perfectly would be, if the eigenvector to it would be a unit-vector along 
\end{enumerate}
\subsection*{Exercise 3}
\begin{enumerate}[(a)]
\item
\item
\item (*bonus)
\end{enumerate}
\end{document}