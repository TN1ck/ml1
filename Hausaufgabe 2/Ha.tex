\documentclass[10pt,a4paper]{article}
\usepackage[utf8]{inputenc}
\usepackage{amsmath}
\usepackage{amsfonts}
\usepackage{amssymb}
\setlength{\parindent}{0cm}
\usepackage{setspace}
\usepackage{mathpazo}
\usepackage{graphicx}
\usepackage{wasysym} 
\usepackage{booktabs}
\usepackage{enumerate}
\usepackage{verbatim}
\usepackage{microtype}
\usepackage{siunitx}
\usepackage{cleveref}
\usepackage[colorlinks=false, pdfborder={0 0 0}]{hyperref}
\usepackage{paralist}
\usepackage{pst-all}
\usepackage{pstricks}
\usepackage{pst-node}
\usepackage{tikz}
\usepackage{tkz-berge}
\usetikzlibrary{trees,petri,decorations,arrows,automata,shapes,shadows,positioning,plotmarks}
\usepackage[a4paper,
left=3.0cm, right=3.0cm,
top=2.0cm, bottom=2.0cm]{geometry}
\usepackage{fullpage}
\usepackage[german]{babel}
%\usepackage{pst-all}
%\usepackage{pstricks}
\setlength{\unitlength}{1cm}
\newcommand{\N}{\mathbb{N}}
\newcommand{\A}{\mathcal{A}}
\newcommand{\R}{\mathbb{R}}
\newcommand{\D}{\mathcal{D}}
\newcommand\numberthis{\addtocounter{equation}{1}\tag{\theequation}}
\usepackage{mathpazo}

\author{Tom Nick, Niklas Gebauer, Felix Bohmann}
\title{Machine Learning}
\begin{document}
\begin{center}
\Large{\textsc{Machine Learning 1: Assignment 2}} \\
\end{center}

\begin{tabbing}
Tom Nick \hspace{0.9cm}\= 340528\\
Niklas Gebauer\>  340942 
\end{tabbing}

\subsection*{Exercise 1}
\begin{enumerate}[(a)]
\item 
With the definition $P(x_k \mid \sigma)$:
$$P(x \mid \sigma) = \begin{cases} \sigma &\text{if} x = \text{head}\quad \\ 1 - \sigma &\text{if}\quad x = \text{tail} \end{cases}$$
We can state the likelihood function $P(\D \mid \sigma)$:
\begin{align*}
	P(\D \mid \sigma) &= \prod_{k = i}^{n} P(x_k \mid \sigma) \\
	&= \sigma^5 \cdot (1 - \sigma)^2
\end{align*}
\item
The maximum likelihood solution for $\sigma$ is simply the sample mean:
$$\sigma = \frac{\#\{ x = \text{head} \mid x \in D\}}{\#(D)} = \frac{5}{7}$$
With this we can compute $P(x_8 = \text{head} ,\quad x_9 = \text{head} \mid \sigma)$:
\begin{align*}
P(x_8 = \text{head} ,\quad x_9 = \text{head} \mid \sigma) = P(x_8 = \text{head} \mid \sigma)P(x_9 = \text{head} \mid \sigma) = \sigma^2 = \frac{25}{49} = 0.51
\end{align*}
\item
With the definition of $p(\sigma) = 1$, the posterior is computed as:
\begin{align*}
p(\sigma \mid \D) &= \frac{p(\D \mid \sigma)p(\sigma)}{p(x)} \\
&= \frac{p(\D \mid \sigma)}{p(x)} \\
&= \frac{\sigma^5 \cdot (1 - \sigma)^2}{p(x)}
\end{align*}

\end{enumerate}






\end{document}